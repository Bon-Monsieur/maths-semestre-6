\documentclass[french,a4paper,10pt]{article}
\input{../common/header.tex}
\usepackage[a4paper,hmargin=30mm,vmargin=30mm]{geometry}

\title{\color{astral} \sffamily \bfseries TD3 - groupes et anneaux 2}
\author{RaphBi}

\begin{document}
    \maketitle
    \begin{td-exo}[1]
        Exprimer le groupe

        $$G = \left\{\begin{pmatrix}
            a & 0 & d\\
            0 & b & 0\\
            0 & 0 & c \end{pmatrix}\ 
            \middle| a,b,c,d \in \bb K, abc=1\right\} < \textmd{GL}_3(\bb K)$$
        comme un produit semi-direct $G = K \rtimes$ H où K $\cong \bb K$ et $H \cong \bb K^\times \times \bb K^\times$.
    \end{td-exo}

    \begin{td-exo}[2]
        Soit $\langle \_,\_\rangle: \bb R^n \rightarrow \bb R$ le produit scalaire standat $\langle x,y\rangle = x^\textmd{T}y$, et soit
        $$\textmd{O}_n = \left\{ A\in \textmd{GL}_n(\bb R)\ | \ \langle Ax,Ay\rangle = \langle x,y \rangle \right\}$$ le groupe orthogonal.
        \begin{enumerate}[($i$)]
            \item Montrer que det $A = \pm 1$ pour tout $A\in \textmd{O}_n$.
            \item Exprimer le groupe $\textmd{O}_n$ comme un produit semi-direct $\textmd{O}_n = \textmd{SO}_n \rtimes H$\\
            où $\textmd{SO}_n = \{A\in \textmd{O}_n \ | \ \textmd{det} \ A = 1\}$ et $H \cong \bm{\mu}_2$.
            \item Montrer que, si $n$ est impair, alors $\textmd{O}_n \cong \ \textmd{SO}_n \times \bm{\mu}_2$.
        \end{enumerate}
    \end{td-exo}

    \begin{td-exo}[3]
        Soit $p$ un nombre premier et soit $0<n<p$ un entier. Montrer que, si $G = K \rtimes H$ est un groupe qui s'écrit comme produit
        semi-direct de deux sous-groupes $K$ et $H$ de cardinal $|K| = n$ et $|H|=p$, alors $G \cong K \times \bm{\mu}_p$.
    \end{td-exo}
    
\end{document}